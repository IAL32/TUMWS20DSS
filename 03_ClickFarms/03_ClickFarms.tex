\section{Grind and Click Farms}
\label{chap:clickfarms}

When I was in high school in Italy, I used to play a free online game called ``Dead Frontier'', a survival MMORPG (Massive Multiplayer Online Role-Playing Games, for you not-nerds in the audience). The goal of the game is simple: shoot at zombies, loot their undead corpses for money or weapons, and level up your character to destroy stronger zombies, get better weapons, and slowly but surely get bad grades at school.

This process is called ``grind'', which basically means repeat the same thing over and over for hundreds of hours. It is a long process, not even so much fun if you ask me now. But indeed, people did it, and still do it today in more modern games. \textit{Dead Frontier}, and other similar games that involve grinding, do offer a simple and expensive way out to get a better account: microtransactions.

Often abbreviated as MTX, microtransactions are a business model where users can purchase virtual goods with micropayments \cite{microtransaction_2021}, and in modern games it's the most widely adopted way to get money out of user's parents credit cards.

In \textit{Dead Frontier}, the maximum achievable level is 220, and according to some of my friends, it takes thousands of hours to get there, even with microtransaction-powered enhancements. Some people then, just go and buy a fully-leveled up account, for hundreds or even thousands of euros, which still won't give your wasted months back.

Ever since Facebook introduced the infamous ``Thumbs Up'' or ``Like'' button on his platform, people immediately saw an opportunity to make money, as it always is. The more likes, views and followers you have on your social media page, the more likely you are to gain trust amongst your users. However, this takes time, effort and a lot of money to pull off. It's a grind.

Exactly like in games, businesses don't always have the time to grow millions of people out of thin air just so they are old enough to hold a smartphone and click ``Like'' on their Facebook page, so they pay what are called ``Click Farms'' to mimic this process, in a shorter amount of time.

Click farms work by buying thousands of smartphones and hiring hundreds of people to do simple tasks, such as to click on their on websites, videos, or following social media accounts. Other than that, they have to do other tasks that don't necessarily have anything to do with their customers, such as googling for some keywords, or scrolling a webpage. This is in fact a crucial point in their work, as these actions prevent their activity to be flagged as potentially malicious or fraudulent. As of today, you can buy roughly ten thousands views on a post on Instagram with as little as 25 dollars. Less if you have a discount code.

As you can imagine, this is a huge problem for both advertisers and the advertising companies. If you are a business trying to sell something online, no matter how much you pay, you will never be able to convert fake likes and clicks into revenue. The data that these fake clicks generate is hugely misleading, and hurtful, as marketing decisions are made through analysis of user's behavior.

However, there are categories of people that base their business entirely on those numbers alone, like influencers or politicians. If you are an average internet user, then the number of likes and followers of an account matters a lot, regardless of what they actually say or do. It's the first impression that matters, and having millions of followers and tens of thousands of likes on content helps that.
