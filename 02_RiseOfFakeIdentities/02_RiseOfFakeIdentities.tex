\section{The Rise of Fake Identities}

Her name is Josie O. Campbell. She lives in Tigard, Oregon, is 45 years old and has been working as a veterinary assistant and laboratory animal caretaker for 10 years. Her car is a 1993 Mazda Lantis, and her favorite color is purple.

The previous paragraph is a description of a fake person, generated by the website \href{https://fakenamegenerator.com}{FakeNameGenerator.com}. It is a rather simple, yet effective free tool that creates a fictional person given a set of basic parameters, such as the gender, name set (American in this case, but it could have been Arabic, Hispanic or even Chinese), and country. In the past years, the need for protecting one's privacy online went up dramatically, aided by the fact that people are increasingly more conscious about their online persona, and the digital trail they might leave behind.

Internet users are using these sorts of fake accounts to sort of ``anonymize'' their presence online, by just hitting the button ``generate'' every time they visit a new website that they don't really care about, or entirely trust.

This situation has caused a lot of trouble amongst social media websites and online forums, as any suspicious or malicious activity is a lot more difficult to tackle with fake information. Because of this, platforms have become more and more active in the pursuit of the deletion of fake accounts. Facebook alone deleted 2.2 \textit{billion} of fake accounts just in the first quarter of 2019, not including the ones that didn't even get activated in the first place.

Most of us who have an account on social medias like Twitter or Instagram, have quite an experience with this kind of fake accounts. For example, you might have received a shady Instagram follow request from a young, prosperous lady with a shadier description, usually with the emoji ``Adults only'', some cucumbers and hearts. Or perhaps you might have heard that your coworker wants to see if his fiancé is loyal, and has created a fake account to go and take a shot to his significant other.

Indeed, there are legitimate uses for fake accounts. To preserve one's identity, to innocently (or not) spying on other people's lives, or to simply please your mother with an Instagram where you just post your best panorama pictures.

However, the problem comes with the fake, automatically created accounts. Their use varies widely, as we previously pointed out, with most of them being used for financial profit. This is not true however for another category of malicious, and potentially more dangerous acts: fake news spreading.

\begin{figure}
    \includegraphics[width=0.5 \linewidth]{02_RiseOfFakeIdentities/image.jpg}
    \caption{Josie O. Campbell: a fake person generated by an AI}
    \label{fig:fakeperson}
\end{figure}
