\section{All Your Data Are Belong To Us}

When the internet started to become mainstream, most people didn't really understand how vulnerable their online presence was, and the same was true the other way around. Webmasters and people who owned the very first servers and windows to the internet, didn't really understand how impactful their user data could be.

The first restaurant to accept online orders was none other than Pizza Hut, in 1994, and it was one of the first 10.000 websites to hit the World Wide Web, and it used an online form to gather their customer's email and phone number information to deliver pizza. Very simple and effective.

Thirty years later, you can create an account, look at a map of all the Pizza Huts in the world, look for a job, download the app, sign up for a fantastic pizza prize. And ah, yes, also order a pizza. In a relatively short amount of time, the number of additional services that website offer skyrocketed, and so has the means of tracking users.

This data is not only used to help users serve a hot, steamy pizza to their doorstep, but also to understand which kind of pizza they want the most, at specific times of the year. This data is used to track and predict where the next Pizza Hut store should be, to maximize the number of pizzas to be delivered.

Some people however, began to be highly concerned with the fact that their presence online was being used not only to satiate their appetite, but also for other means. Websites like \textit{Google} and \textit{Yahoo!} saw the potential of using user's searches through the web to show them advertisements. So suddenly, after you visit \textit{PizzaHut.com} for your midnight cravings, the next time you visit \textit{Bloomberg.com} a small box pops out, with \textit{Domino's} menu.

People grew concerned and, as it always was, wanted to regain their freedom of choosing what their information was being used for.
