\section{Fake News}

You might have subscribed to online news magazines from time to time, and read through their articles. They might not look different at all from the ones that we see in real life, except from the fact that they are written on pixels on screen instead of ink on paper. Even the titles sound the same: ``Godzilla Invaded New York!'', ``President Shaking Hands With Pope'', ``Winter Is Coming: Buy BrandName''.

However, titles, \textit{recommendations} that you see and the articles that you read, evolved along with the algorithms that show them to you. This became overtime one of the most debated arguments of how the internet works, and how companies make money. On YouTube for example, content creators have to keep track of every algorithm change, leading to the fact that everything they create, is not actually done to appease the users, but to appease the algorithm. If you don't show in people's recommendations, you don't exist.

Titles, thumbnails, video length, sponsors, advertisements. Everything has to be done in such a way that the algorithm favors your videos amongst a million other. Of course, this also applies everywhere a content creator posts their content, such as Google, Instagram, LinkedIn or Twitter. This is why it is more and more common to see titles such as: ``HOW TO BECOME A BIRD IN 10 STEPS'', ``JOKE: BIG BALL HITS PEOPLE, PRANK!'' and thumbnails with bright, saturated pictures of people with their mouths open in awe, and some dollar bill signs, because why not.

Of course, this is not entirely to blame to algorithms. We as users click on those articles and videos, basically telling the underlying system that yes, this system works, and we want to see that video and watch it to the end. This is how the internet uses algorithms to hack human nature.

This ``hack'' can in turn be used to manufacture articles in such a way that people are guaranteed to take a look at them, starting from the title. As an example, using ``Pope Francis Meets Donald Trump'' is much less effective in gaining clicks than ``Pope Francis Shocks The World, Endorses Donald Trump for President''. Although untrue, the latter is far more likely to be seen in social media, to get clicked on and to get shared amongst peers. In turn, this generates revenue for the article poster, via clicks on advertisement banners, which usually proliferate on their websites like shrooms in a forest.

You might have had the misfortune to find somebody in your life that shared some of these kind ``clickbaity'' of articles to you, or to their social media pages, like their \textit{wall} on Facebook or as a \textit{tweet} on Twitter. What I call misfortune is confronting them on the truthfulness of said article, and failing to making their posters to admit that they were wrong about believing that Elizabeth, Queen of England, is actually just a pawn of a secret group called the ``World Order''.

What happened there? Why did your friend start to believe in such stories, and why? Well, as it always has been, when enough people start to believe in something, that belief becomes reality. An untrue one, but an alternative reality nonetheless.

The same thing can be said for example about Galileo Galilei, the scientist who in 1609 defended his position of heliocentrism based on his astronomical observations in front of a jury of a very Christian jury. At that time, the occidental world lived within an alternative reality, where the earth is at the center of the universe, and the Earth was flat.

This happened because enough people believed that heliocentrism was a foolish idea, based on the Bible.

What if the same thing happened today? We are in an era when ideas and opinions can go far and wide, in the blink of an eye. News, fake or true, can be told by anyone anywhere, in any position of power, for whatever reason.

Election results can suddenly change with a few tweets, trends emerge with a few viral videos on TikTok, and vaccines and a global pandemic that kills hundreds of thousands of people suddenly becomes a hoax.
